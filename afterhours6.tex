% afterhours6.tex - Afer Hours Week 6
\chapter{After Hours Week 6}
\section{``Tug of war''}
\subsection{Taking the push with the pull}

We have spoken in fairly great length about how remote repositories work.  We have seen how the \texttt{git remote} tool is used to create the various references to remote repositories, but we have no real understanding about what this means in terms of Git's internals.  Just in the same way a branch is a single file that contains a pointer to a reference, a remote repository has to be handled within Git somehow.

As it happens, Git again uses a reasonably simplistic design when creating remote references.  To take a look at this in detail, we need to once again delve into the \texttt{.git} directory.  Seeing as our original repository does not contain any remotes for now, we are going to use our \texttt{coderepo-cl} folder as an example.  Hopefully, if you have been following the text, you have not deleted this directory yet.  If you have, do not worry, just follow the operations we completed in Week 6, or read on and use the text in the book.

If you remember, we created two clones of our original repository.  Once was a simple clone called \texttt{coderepo-cl} and the other was a bare repository called \texttt{coderepo-bk}.  The \texttt{coderepo-cl} and the \texttt{coderepo-bk} repositories were both cloned from \texttt{coderepo}, but it was \texttt{coderepo-cl} that was configured to pull from one and push to the other.  Running a simple \texttt{git remote -v} command, confirms this configuration.

\begin{Verbatim}
john@satsuki:~/coderepo-cl$ git remote -v
backup	/home/john/coderepo-bk (fetch)
backup	/home/john/coderepo-bk (push)
origin	/home/john/coderepo (fetch)
origin	/home/john/coderepo (push)
john@satsuki:~/coderepo-cl$ 
\end{Verbatim}

We can get even more information by running the \texttt{git remote show} tool with the remote name as a parameter.

\begin{Verbatim}
john@satsuki:~/coderepo-cl$ git remote show origin
* remote origin
  Fetch URL: /home/john/coderepo
  Push  URL: /home/john/coderepo
  HEAD branch: master
  Remote branches:
    master    tracked
    wonderful tracked
    zaney     tracked
  Local branches configured for 'git pull':
    master    merges with remote master
    wonderful merges with remote wonderful
  Local refs configured for 'git push':
    master    pushes to master    (up to date)
    wonderful pushes to wonderful (up to date)
john@satsuki:~/coderepo-cl$ 
\end{Verbatim}

How though, is this data set up and configured from within Git itself.  Looking at the \texttt{.git/config} file, we can see a glimpse of this.

\begin{Verbatim}
john@satsuki:~/coderepo-cl$ cat .git/config 
[core]
	repositoryformatversion = 0
	filemode = true
	bare = false
	logallrefupdates = true
[remote "origin"]
	fetch = +refs/heads/*:refs/remotes/origin/*
	url = /home/john/coderepo
[branch "master"]
	remote = origin
	merge = refs/heads/master
[branch "wonderful"]
	remote = origin
	merge = refs/heads/wonderful
[remote "backup"]
	url = /home/john/coderepo-bk
	fetch = +refs/heads/*:refs/remotes/backup/*
john@satsuki:~/coderepo-cl$ 
\end{Verbatim}

As you can see
