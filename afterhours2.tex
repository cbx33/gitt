% afterhours2.tex - Afer Hours Week 2
\chapter{After Hours Week 2}
\section{``A Litte Of Git's Internals''}
\subsection{A Look At Plumbing}

We are going to use the Git repository that we have been playing with in Week 2 and start to take a deeper look at what is actually inside a Git repository.  To begin with, let's take a brief look into the directory structure, to see what has been created with a simple \texttt{git init}  command in the \texttt{.git} folder.

\begin{Verbatim}[frame=leftline,framerule=1mm,fontsize=\relsize{-3}] 
john@akira:~/coderepo/.git$ ls -la 
total 40 
drwxr-xr-x 7 john john 4096 2011-02-17 19:23 . 
drwxr-xr-x 3 john john 4096 2011-02-17 19:23 .. 
drwxr-xr-x 2 john john 4096 2011-02-17 19:23 branches 
-rw-r--r-- 1 john john   92 2011-02-17 19:23 config 
-rw-r--r-- 1 john john   73 2011-02-17 19:23 description 
-rw-r--r-- 1 john john   23 2011-02-17 19:23 HEAD 
drwxr-xr-x 2 john john 4096 2011-02-17 19:23 hooks 
drwxr-xr-x 2 john john 4096 2011-02-17 19:23 info 
drwxr-xr-x 4 john john 4096 2011-02-17 19:23 objects 
drwxr-xr-x 4 john john 4096 2011-02-17 19:23 refs 
john@akira:~/coderepo/.git$ 
\end{Verbatim} 

\textbf{branches} - Though deprecated now, this folder stores shorthands for git pull, push and fetch commands, by creating a file, the name of which is passed to the command instead of the repository argument.

\textbf{config} - This is the main configuration file for Git.  It is the first place git looks for upon invocation.  If this file is not present, Git will inspect ~/.gitconfig.  After this, Git will go to /etc/gitconfig.  The file holds information about the remotes, tracking branches, push configurations and many more items.

\textbf{description} - This is a simple text file which gives a description to a repository when being view via gitweb or similar.

\textbf{HEAD} - This file is a pointer to the parent commit of your current branch.

\textbf{hooks} - Scripts can be placed in here to perform operations at certain points during the commit process.

\textbf{info} - The info folder contains some additional information about the repository

\textbf{objects} - The is the directory that holds all of the actual files that are stored in the repository.  The files are named by their SHA-1 values.  Inside the folder are a number of directories which make up the first 2 characters of the SHA-1 value.  The remaining portion of the SHA-1 hash is used to name the file.

\textbf{refs} - This folder holds the files that files for local branches, remote branches and tags.

More files and folders will appear here during the running of the repository as you begin to start using different features in Git.
